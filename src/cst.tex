\documentclass[a4paper]{article}

% natbib
\usepackage[numbers,sort]{natbib}

\usepackage{mysimpletemplatewn}
\usepackage[utf8]{inputenc} %%% to support copy and paste with accents for french stuff
\usepackage[T1]{fontenc} %%%key to get copy and paste for the code!
\usepackage{times}
\usepackage[french]{babel}
\usepackage[scaled=0.85]{helvet}
\usepackage{graphicx}
\usepackage{ifthen}
\usepackage{xspace}
\usepackage{alltt}
\usepackage{latexsym}
\usepackage{url}
\usepackage{amsmath,amssymb,amsfonts}
\usepackage{stmaryrd}
\usepackage{algorithmic}
\usepackage{textcomp}
\usepackage{xcolor}
\usepackage{enumerate}
% \usepackage{cite}
\usepackage[pdftex,colorlinks=true,pdfstartview=FitV,linkcolor=blue,citecolor=blue,urlcolor=blue]{hyperref}
\usepackage{multirow}
\usepackage{listings}
\usepackage{color}
\usepackage{subcaption}

\usepackage{pgfgantt}

\usepackage{bibentry}
\nobibliography*


\newboolean{showcomments}
\setboolean{showcomments}{true}
\ifthenelse{\boolean{showcomments}}
  {\newcommand{\bnote}[2]{
	\fbox{\bfseries\sffamily\scriptsize#1}
    {\sf\small$\blacktriangleright$\textit{#2}$\blacktriangleleft$}
    % \marginpar{\fbox{\bfseries\sffamily#1}}
   }
   \newcommand{\paragraphDesc}[2]{
    \textcolor{#1}{#2}
   }
   
  }
  {\newcommand{\bnote}[2]{}
   \newcommand{\cvsversion}{}
   \newcommand{\paragraphDesc}[2]{}
  }

\newcommand{\here}{\bnote{***}{CONTINUE HERE}}
\newcommand{\nb}[1]{\bnote{NB}{#1}}

\newcommand{\fix}[1]{\bnote{FIX}{#1}}
%%%% add your own macros 


\newcommand{\sd}[1]{\bnote{Stef}{\textcolor{orange}{#1}}}
\newcommand{\an}[1]{\bnote{Anne}{\textcolor{green}{#1}}}
\newcommand{\bv}[1]{\bnote{Benoit}{\textcolor{blue}{#1}}}
\newcommand{\nic}[1]{\bnote{Nic}{\textcolor{red}{#1}}}

\newcommand{\todo}[1]{\bnote{TODO}{#1}}

\graphicspath{{figures/}}
%%% 


\newcommand{\figref}[1]{Figure~\ref{fig:#1}}
\newcommand{\figlabel}[1]{\label{fig:#1}}
\newcommand{\tabref}[1]{Table~\ref{tab:#1}}
\newcommand{\layout}[1]{#1}
\newcommand{\commented}[1]{}
\newcommand{\secref}[1]{Section~\ref{sec:#1}}
\newcommand{\seclabel}[1]{\label{sec:#1}}

%\newcommand{\ct}[1]{\textsf{#1}}
\newcommand{\stCode}[1]{\textsf{#1}}
\newcommand{\stMethod}[1]{\textsf{#1}}
\newcommand{\sep}{\texttt{>>}\xspace}
\newcommand{\stAssoc}{\texttt{->}\xspace}

\newcommand{\stBar}{$\mid$}
\newcommand{\stSelector}{$\gg$}
\newcommand{\ret}{\^{}}
\newcommand{\msup}{$>$}
%\newcommand{\ret}{$\uparrow$\xspace}

\newcommand{\myparagraph}[1]{\noindent\textbf{#1.}}
\newcommand{\eg}{\emph{e.g.}\xspace}
\newcommand{\ie}{\emph{i.e.}\xspace}
\newcommand{\etal}{\emph{et al.,}\xspace}
\newcommand{\ct}[1]{{\textsf{#1}}\xspace}
\newcommand{\etc}[1]{\textit{etc.}}



\newcommand{\defaultScale}{0.55}
\newcommand{\pic}[3]{
   \begin{figure}[h]
   \begin{center}
   \includegraphics[scale=\defaultScale]{#1}
   \caption{#2}
   \label{#3}
   \end{center}
   \end{figure}
}


\newcommand{\twocolumnpic}[3]{
  \begin{figure*}[!ht]
  \begin{center}
  \includegraphics[scale=\defaultScale]{#1}
  \caption{#2}
  \label{#3}
  \end{center}
  \end{figure*}
}

\newcommand{\picw}[3]{
  \begin{figure}[htbp]
  \begin{center}
  \includegraphics[width=\columnwidth]{#1}
  \caption{#2}
  \label{#3}
  \end{center}
  \end{figure}
}


\newcommand{\infe}{$<$}
\newcommand{\supe}{$\rightarrow$\xspace}
\newcommand{\di}{$\gg$\xspace}
\newcommand{\adhoc}{\textit{ad-hoc}\xspace}

\usepackage{url}            
\makeatletter
\def\url@leostyle{%
  \@ifundefined{selectfont}{\def\UrlFont{\sf}}{\def\UrlFont{\small\sffamily}}}
\makeatother
% Now actually use the newly defined style.
\urlstyle{leo}

\definecolor{codegreen}{rgb}{0,0.6,0}
\definecolor{codegray}{rgb}{0.5,0.5,0.5}
\definecolor{codepurple}{rgb}{0.58,0,0.82}
\definecolor{backcolour}{rgb}{0.95,0.95,0.92}

\lstdefinestyle{mystyle}{
    backgroundcolor=\color{backcolour},
    commentstyle=\color{codegreen},
    keywordstyle=\color{magenta},
    numberstyle=\tiny\color{codegray},
    stringstyle=\color{codepurple},
    basicstyle=\footnotesize,
    breakatwhitespace=false,
    breaklines=true,
    captionpos=b,
    keepspaces=true,
    numbers=left,
    numbersep=5pt,
    showspaces=false,
    showstringspaces=false,
    showtabs=false,
    tabsize=2
}

\lstset{style=mystyle}


\title{CST --Analyse multi-facettes et opérationnelle pour la transformation des systemes d’information}
\author{ HOUEKPETODJI Mahugnon Honoré}

\begin{document}

\institution{}

\date{\today}

\maketitle

% \begin{abstract}
% The abstract text goes here.
% \end{abstract}

\section{Description du sujet de la thèse}
% 0.5 (1/2) pages
CIM est une SAS au capital social de 200k détenu a 100 par DL Software. 
CIM est éditeur, intégrateur, hébergeur et infogéreur de solutions pour l'assurance de personnes en santé, prévoyance. 
Elle offre une expertise Santé et Prévoyance acquise après
plus de 30 ans auprès de ses clients. 
CIM est hébergeur de ses solutions pour 90 de ses clients et plus de 1000 utilisateurs. 
Toutes les thématiques d'infrastructure et de surveillance des flux sont intégrées à cette offre.
CIM est propriétaire de ses infrastructures serveurs, tous les éléments actifs des systèmes et tous les éléments de stockage sont achetés par CIM, gérés et supervisés par les équipes de CIM. Aucun sous-traitant n'intervient dans les opérations quotidiennes d'hébergement, d'exploitation des solutions et des données hébergées.

CIM est certifiée Microsoft GOLD Partner. 
Elle est l'éditeur des progiciels de la gamme Izy Links et assure l'intégration de l'ensemble des briques de cette gamme ainsi que des briques partenaires nécessaires à la bonne réussite du projet. Cette solution est développée en PowerBuilder sur base de données DB2. L'équipe de développement
vient d'upgrader en PowerBuilder 2017 (été 2018) et est en cours de passage sur DB2 v11 (avec l'aide d'un DBA IBM – prestation 2018).

Le système de gestion est centré sur le back office Izy Protect, autour duquel gravite
l’ensemble des briques complémentaires répondant à l'assemble des besoins, et pouvant
être activées ou non. 
La société CIM a effectué une analyse de risque pour son évolution et croissance en 2017 d'où il ressort que Izy Protect souffre des problèmes suivant:
\begin{itemize}
\item vieux langage  
\item logiciel vieillissant
\item perte savoir
\item changements à haut risque 
\end{itemize}
Ces problèmes sont récurrents chez les organises gérant des systèmes d'information \cite{Deme02a}.

Ce travail de doctorat consiste de proposer des modèles et des mécanismes permettant d'assurer
une régénération des systèmes d'information. Les expériences et validation des prototypes se feront dans le contexte de l'application du système d'information écrit en PowerBuilder de la société CIM
\section{Etat de l'art}
\label{sec:stateOfTheArt}
% 1 pages

La~\secref{migrationTechnique} présente les techniques utilisées pour migrer une application.
\secref{position}, nous décrivons les méta-modèles d'interface graphiques trouvés dans la littérature.

\subsection{Stratégie de migration existantes}
\label{sec:migrationTechnique}

\subsection{Représentation de l'interface utilisateur}
\label{sec:position}

\subsubsection{Les standards de l'OMG}
\label{sec:omg}

\subsubsection{méta-modèles GUI dans la littérature}
\label{sec:stateMetaUI}

\section{Avancées actuelles}
% 3 pages


\subsection{Context}
\label{sec:context}

\subsubsection{Comparison de GWT et Angular}
\label{sec:comparisonGwtAngular}
\subsubsection{Structure d'application front-end}
\label{sec:guiDecomposition}


\subsection{Approche}
\label{sec:approche}

\subsubsection{Processus de migration}
\label{sec:processusMigration}
\subsubsection{Méta-modèle de layout}
\label{sec:layout}
\subsection{Premier résultats}
\label{sec:resultats}

\subsubsection{Application industrielle}
\label{sec:industrial}


\subsubsection{Résultats de l'extraction}
\label{sec:retroImport}

\subsubsection{Résultats à l'exportation}
\label{sec:export}


\subsubsection{Collaborations}
\label{sec:impact}

\section{Roadmap}
\label{sec:roadmap}

\subsection{Migration Multi-Langage}
\label{sec:migrationMultiLangage}

\subsection{Code comportemental}
\label{sec:codeComportemental}

\subsection{Code métier}
\label{sec:codeMetier}

\subsection{Validation automatique}
\label{sec:validationAutomatique}

\subsection{Rédaction de thèse}
\label{sec:redaction}


\section{Publications}

Cette section présente la liste des publications et soumissions liée à cette thèse.

\subsection{Papiers publiés}

\begin{enumerate}
  \item Conférence internationale : \bibentry{Verh19a}
  \item Conférence nationale : \bibentry{Verh19b}
  \item Workshop internationale : \bibentry{Verh19c}
\end{enumerate}

\subsection{Papiers soumis}

\begin{enumerate}
  \item Workshop internationale : \bibentry{Verh19d}
  \item Conférence internationale de rang C : \bibentry{Verh19e}
\end{enumerate}

\section{Formation doctorale}


\section{Projet professionnel}



\footnotesize{
  %\bibliographystyle{alpha}
  % natbib 
  %\bibliographystyle{myplainnat}
  \bibliographystyle{mySmallPlainnat}
  \bibliography{others,rmod,nextPubli}
}

\end{document}